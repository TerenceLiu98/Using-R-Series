\documentclass[10pt,aspectratio=32]{beamer}
\usepackage{sgp940}

\title{Box-Tidwell Transformation}
\subtitle{Realization with R language}
\vspace{1.5cm}
\author{Yi He \and ChengYu Huang \and JunJie Liu \\ \and JiaCheng Tu \and DaiChen Yao }
\institute{United International College - STAT}
\date{\today}

\begin{document}
\AtBeginSection[]{\frame{\sectionpage}} %insert section title pages (new frames)

\begin{frame}
	\titlepage
\end{frame}

\begin{frame}
	\frametitle{Overview}
	\vspace{-0.7cm}
	\tableofcontents
\end{frame}

\section{Intro}
\subsection{Background}

%STEP BY STEP PROCEDURE
\begin{frame}
\frametitle{Step by Step Procedure}
\vspace{-0.3cm}

\textbf{Step Zero}

$$\begin{align}
&y = \beta_0 + \beta_1 w_1 + \dots + \beta_k w_k + \epsilon \label{eq:ordinary linear regression} \\
&w_j = f(x) =\left\{
\begin{aligned}
&x_j^{\alpha_j}  \alpha \neq 0 \\
&ln(x_j), \alpha = 0 \label{eq:ransformation of X}\\
\end{aligned}
\right.
\end{align}
$$
The method accommodates exponents on one or more of the regressor variables. $\alpha_1, \alpha_2, \dots,\alpha_k $
 \end{frame}

\begin{frame}
	\frametitle{Step by Step Procedure (Continued)}
\textbf{Step One (Inital Model)}

We can get $W = X^{\alpha} \Rightarrow w_i = x_i^{\alpha_i}$, note that  $\underbrace{\alpha_1, \alpha_2, \alpha_3, \dots, \alpha_k}_{\text{The inital $\alpha$}} = 1$, we can get that $W = X ^ alpha = X^1$ since Equation \ref{eq:ordinary linear regression} \\ the \textbf{new regression model is}:
$$y = \beta_0 + \beta_1 w_1^* + \dots + \beta_k w_k^* \Rigtharrow \hat{beta_0}, \hat{\beta_1}, \dots, \hat{\beta_k} \Rightarrow \beta_{out} = (\beta_0, \beta_1, \beta_k)_{1 \times (k + 1))}$$
\end{frame}

\begin{frame}
 	\frametitle{Step by Step Procedure (Continued)}
 \textbf{Step Two (New Model)}

Then, We add $Z$ to the equation as $z_i = w_i ln(w_i)$ and we can get the new $\beta$ and $\gamma$:

$$\begin{align*}
&y = \beta_0 + \beta_1 w_1^* + \dots + \beta_k w_k^* + \gamma_1 z_1^* \gamma_2 z_2^* + \dots + \gamma_k z_k^* \Rightarrow \hat{\gamma_1}, \hat{\gamma_2}, \dots, \hat{\gamma_k} \label{eq:new linear regression model}\\
&\beta_{out} = (\beta_0', \beta_1', \dots, \beta_k, \hat{\gamma_1}, \hat{\gamma_2}, \dots, \hat{\gamma_k}) \label{eq:new model coefficient}
\end{align*}$$

\end{frame}

\begin{frame}
 	\frametitle{Step by Step Procedure (Continued)}
 \textbf{Step Two (Update Term)}

 We compute the update $\hat{alpha_j}$ with the formula:

 $$
 	\hat{\alpha_j} = (\frac{\hat{\gamma}_j}{\hat{\beta_j}} + 1) \times (\text{ Current Value of } \hat{\alpha_j})
 $$

We can get the new $\alpha$ for the new iteration, after $n$ times iterations $(|(\alpha_{k-1} - \alpha_{k-2}| \leq \epsilon$,$\epsilon$ is the tolrance value), we can stop the iteration and at that time, the $\alpha$ is what we need: the relative best fitting curve. In multiple regression it can also be a good use of this method, one more variable is to do a set of data loop.
\end{frame}


\section{Motivation}
\begin{frame}
	\frametitle{Paragraphs of Text}
	\vspace{-0.8cm}
	Sed iaculis dapibus gravida. Morbi sed tortor erat, nec interdum arcu. Sed id lorem lectus. Quisque viverra augue id sem ornare non aliquam nibh tristique. Aenean in ligula nisl. Nulla sed tellus ipsum. Donec vestibulum ligula non lorem vulputate fermentum accumsan neque mollis.\\~\\

	Sed diam enim, sagittis nec condimentum sit amet, ullamcorper sit amet libero. Aliquam vel dui orci, a porta odio. Nullam id suscipit ipsum. Aenean lobortis commodo sem, ut commodo leo gravida vitae. Pellentesque vehicula ante iaculis arcu pretium rutrum eget sit amet purus. Integer ornare nulla quis neque ultrices lobortis. Vestibulum ultrices tincidunt libero, quis commodo erat ullamcorper id.
\end{frame}



\section{Model}


\begin{frame}[fragile] % Need to use the fragile option when verbatim is used in the slide
\frametitle{Verbatim}
\begin{example}[Theorem Slide Code]
\begin{verbatim}
\begin{frame}
\frametitle{Theorem}
\begin{theorem}[Mass--energy equivalence]
$E = mc^2$
\end{theorem}
\end{frame}\end{verbatim}
\end{example}
\end{frame}


\begin{frame}
	\frametitle{\TeX{}}
	\begin{columns}
		\begin{column}{4cm}
			\begin{figure}
				\includegraphics[height=3cm]{cuba_logo}
			\end{figure}
			\begin{center}
				\tiny
				Department of Decision Sciences and\
				Managerial Economics \\
				The Chinese University of Hong Kong \\
			\end{center}
		\end{column}
		\begin{column}{6cm}
			\begin{itemize}
				\item \TeX{} was created by Donald Knuth in 1978
				\item A typesetting macro language and compiler:
				\begin{itemize}
					\item Readable mathematics
					\item Better hyphenation
					\item Optimized justification
					\item Font management tools
					\item Cross-compatibility
				\end{itemize}
				\item Code -- Compile -- Visualize
			\end{itemize}
		\end{column}
	\end{columns}
\end{frame}



\section{Data and Method}

\begin{frame}
 \frametitle{Figure}
 \vspace{-0.3cm}
	\begin{figure}[h]
		\centering
		\includegraphics[width=0.3\textwidth]{cuba_logo}
		\caption{CUHK Business School}
	\end{figure}
\end{frame}


\begin{frame}
	\frametitle{Editors and Compilers}
	\begin{itemize}
		\item To install in your machine
		\begin{itemize}
			\item Check \texttt{latex-project.org}
		\end{itemize}
		\item In the cloud
		\begin{itemize}
			\item ShareLatex : \texttt{www.sharelatex.com}
			\item Overleaf : \texttt{www.overleaf.com}
		\end{itemize}
	\end{itemize}
	\vskip 1cm
	\begin{block}{Please give me Mb of space on Overleaf}
		https://www.overleaf.com/signup?ref=d1806010dac8
	\end{block}
\end{frame}


\begin{frame}
	\frametitle{Multiple Columns}
	\begin{columns}[c]
		\column{.45\textwidth} % Left column and width
		\textbf{Heading}
		\begin{enumerate}
			\item Statement
			\item Explanation
			\item Example
		\end{enumerate}
		\column{.5\textwidth} % Right column and width
		Lorem ipsum dolor sit amet, consectetur adipiscing elit. Integer lectus nisl, ultricies in feugiat rutrum, porttitor sit amet augue. Aliquam ut tortor mauris. Sed volutpat ante purus, quis accumsan dolor.
	\end{columns}
\end{frame}



\section{Conclusion}

\begin{frame}
\frametitle{Table and Equation}
\vspace{-0.3cm}
	\begin{table}
		\begin{tabular}{l l l}
			\toprule
			\textbf{Treatments} & \textbf{Response 1} & \textbf{Response 2}\\
			\midrule
			Treatment 1 & 0.0003262 & 0.562 \\
			Treatment 2 & 0.0015681 & 0.910 \\
			Treatment 3 & 0.0009271 & 0.296 \\
			\bottomrule
		\end{tabular}
		\caption{Table caption}
	\end{table}

\begin{equation} % use equation* to remove equation number
\label{eq:matrix_transpose}
\left[
\begin{array}{ccc}
a_{11} & \cdots & a_{1n} \\
\vdots & \ddots & \vdots \\
a_{n1} & \cdots & a_{nn}
\end{array}
\right]^T
=
\left[
\begin{array}{ccc}
a_{11} & \cdots & a_{n1} \\
\vdots & \ddots & \vdots \\
a_{1n} & \cdots & a_{nn}
\end{array}
\right]
\end{equation}
\end{frame}


\begin{frame}
	\frametitle{References}
	\footnotesize{
		\begin{thebibliography}{99} % Beamer does not support BibTeX so references must be inserted manually as below
			\bibitem[Smith, 2012]{p1} John Smith (2012)
			\newblock Title of the publication
			\newblock \emph{Journal Name} 12(3), 45 -- 678.
		\end{thebibliography}
	}
\end{frame}


\begin{frame}
 	\titlepage
\end{frame}

\end{document}
